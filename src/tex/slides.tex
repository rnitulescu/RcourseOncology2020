%%%%%%%%%%%%%%%%%%%%%%%%%%%%%%%%%%%%%%%
%% Slides : R for clinical trialists %%
%%%%%%%%%%%%%%%%%%%%%%%%%%%%%%%%%%%%%%%

%% PREAMBLE
%% Define document class and basic options
\documentclass{beamer}
%\setlength{\parindent}{0pt}

%% Load packages
\usepackage{palatino}
%\usepackage{amsfonts}
%\usepackage{amsmath}
%\usepackage{listings}
\usepackage{verbatim}
\usetheme{Madrid}

%% Basic info
\title{R for Clinical Trialists}
\subtitle{An introduction}
\author{Roy Nitulescu\inst{1}}

\institute
{
    \inst{1}%
    CITADEL\\
    CR-CHUM
}

\date{McGill University, Nov. 18, 2020}

\AtBeginSection[]
{
    \begin{frame}
        \frametitle{Table of Contents}
        \tableofcontents[currentsection]
    \end{frame}
}


%% BEGIN DOCUMENT
\begin{document}

%%%%
%% Slides
%%%%

\frame{\titlepage}

\begin{frame}
    \frametitle{Table of Contents}
    \tableofcontents
\end{frame}


%%%%
%% MODULE 1
%%%%

\section{Module 1: Introduction to R}

\subsection{Brief overview of R programming language}

%% Intro
\begin{frame}
    \frametitle{What is R?}
    \begin{enumerate}
      \item Free/Libre and Open Source Software (FLOSS)
      \begin{enumerate}
        \item Free as in ``free beer'', but also free as in ``free speech''
        \item Users are free to look at the source code, suggest improvements, submit bug fixes, and create custom libraries
        \item This encourages the development of a community of users working together to improve the software and make it accessible
      \end{enumerate}
      \item R is a software environment for statistical computing and graphics
      \begin{enumerate}
        \item Process data, calculate summary statistics, and fit multivariate models with very litte code
        \item Create standard plots with very little code or build custom graphics to suit any needs
        \item Interact with your operating system to automate many tedious tasks
      \end{enumerate}
    \end{enumerate}
\end{frame}


\begin{frame}
    \frametitle{Strengths and Weaknesses}
    \begin{enumerate}
      \item Strengths
      \begin{enumerate}
        \item Accessible, portable, and well documented
        \item Flexible and extendable
        \item Relatively efficient (especially compared to SAS)
      \end{enumerate}
      \item Weaknesses
      \begin{enumerate}
        \item Data must fit in memory (this can be overcome with some libraries)
        \item Steep learning curve (especially compared to SAS or Stata)
        \item Efficient extension requires professional programming know-how
      \end{enumerate}
    \end{enumerate}
\end{frame}


%% Core material
\subsection{Basic R objects}

\begin{frame}[fragile]
    \frametitle{The environment}
    \begin{enumerate}
      \item Working directory, paths, files
      \item libraries
      \item Objects in the local environment
      \item \texttt{options()} function (see documentaton for details)
    \end{enumerate}
    \verbatiminput{../R/module1/environment.Rout}
\end{frame}


\begin{frame}[fragile]
    \frametitle{Scalars and vectors}
    \verbatiminput{../R/module1/scalars_vectors.Rout}
\end{frame}


\begin{frame}[fragile]
    \frametitle{Matrices}
    \verbatiminput{../R/module1/matrices.Rout}
\end{frame}


%%%%
%% MODULE 2
%%%%

\section{Module 2: Data analysis in R}



%%%%
%% MODULE 3
%%%%

\section{Module 3: Research design in R}




%% END DOCUMENT
\end{document}

