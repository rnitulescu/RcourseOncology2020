%%%%%%%%%%%%%%%%%%%%%%%%%%%%%%%%%%%%%%%
%% Slides : R for clinical trialists %%
%%%%%%%%%%%%%%%%%%%%%%%%%%%%%%%%%%%%%%%

%% PREAMBLE
%% Define document class and basic options
\documentclass{beamer}
%\setlength{\parindent}{0pt}

%% Load packages
\usepackage{palatino}
%\usepackage{amsfonts}
%\usepackage{amsmath}
%\usepackage{url}
\usepackage{hyperref}
%\usepackage{listings}
\usepackage{verbatim}

\hypersetup{
	colorlinks=true,
	linkcolor=blue,
	citecolor=red,
	filecolor=blue,
	urlcolor=blue
}

\usetheme{Madrid}

%% Basic info
\title{R for Clinical Trialists}
\subtitle{An Introduction}
\author{Roy Nitulescu\inst{1}}

\institute
{
    \inst{1}%
    CITADEL\\
    CR-CHUM
}

\date[McGill, Nov. 18, 2020]{McGill University, Nov. 18, 2020}

\AtBeginSection[]
{
    \begin{frame}
        \frametitle{Table of Contents}
        \tableofcontents[currentsection]
    \end{frame}
}

\AtBeginSubsection[]
{
    \begin{frame}
        \frametitle{Table of Contents}
        \tableofcontents[currentsubsection]
    \end{frame}
}


%% BEGIN DOCUMENT
\begin{document}

%%%%
%% Slides
%%%%

\frame{\titlepage}


\begin{frame}
    \frametitle{Preamble}
    \begin{itemize}
      \item This 3-hour lecture will be broken down into 3 modules
      \item Each module will last 50 minutes
      \begin{itemize}
        \item 30 minutes of lecture time
        \item 10 minutes for exercises
        \item 10 minutes for discussing the exercises
      \end{itemize}
      \item There will be two 15-minute breaks between the modules
      \item I expect that all students have already installed R on their computer and tested that it works
    \end{itemize}
    
    \vfill
    
    The source code for this presentation can be found here:\\
    \url{https://github.com/rnitulescu/RcourseOncology2020}
\end{frame}


\begin{frame}
    \frametitle{Table of Contents}
    \tableofcontents
\end{frame}


%%%%
%% MODULE 1
%%%%

\section{Module 1: Introduction to R}

\subsection{Getting started}

%% Intro
%% Commented out this and the next slide and replaced with one simpler slide
%\begin{frame}
%    \frametitle{What is R?}
%    \begin{enumerate}
%      \item Free/Libre and Open Source Software (FLOSS)
%      \begin{enumerate}
%        \item Free as in ``free beer'', but also free as in ``free speech''
%        \item Users are free to look at the source code, suggest improvements, submit bug fixes, and create custom libraries
%        \item This encourages the development of a community of users working together to improve the software and make it accessible
%      \end{enumerate}
%      \item R is a software environment for statistical computing and graphics
%      \begin{enumerate}
%        \item Process data, calculate summary statistics, and fit statistical models with very litte code
%        \item Create standard plots with very little code or build custom graphics to suit any needs
%        \item Interact with your operating system to automate many tedious tasks
%      \end{enumerate}
%    \end{enumerate}
%\end{frame}


%\begin{frame}
%    \frametitle{Strengths and Weaknesses}
%    \begin{enumerate}
%      \item Strengths
%      \begin{enumerate}
%        \item Accessible, portable, and well documented
%        \item Flexible and extendable
%        \item Relatively efficient (especially compared to SAS)
%      \end{enumerate}
%      \item Weaknesses
%      \begin{enumerate}
%        \item Data must fit in memory (this can be overcome with some libraries)
%        \item Steep learning curve (especially compared to SAS or Stata)
%        \item Efficient extension requires professional programming know-how
%      \end{enumerate}
%    \end{enumerate}
%\end{frame}


%% Simpler version of the above
\begin{frame}
    \frametitle{What is R?}
    \begin{itemize}
	  \item Software environment for statistical computing and graphics
	  \item Steep learning curve
	  \item Worth learning due to its ubiquity, versatility, and efficiency
    \end{itemize}
\end{frame}


\begin{frame}
	\frametitle{Installing R}
	\begin{itemize}
	  \item Web page: \url{http://cran.utstat.utoronto.ca/}\footnote{This is the Ontario mirror. You can choose any mirror from here: \url{https://cran.r-project.org/mirrors.html}}
	  \item \textbf{Windows and Mac}: Download and install precompiled binary distributions
	  \item \textbf{Linux}: Install through package manager or compile from source
	\end{itemize}
\end{frame}


\begin{frame}
    \frametitle{Getting help}
    \begin{itemize}
	  \item Official documentation: \url{https://cran.r-project.org/manuals.html}
	  \item Cheat sheet: \href{https://cran.r-project.org/doc/contrib/Short-refcard.pdf}{Link}
	  \item Forums: \url{https://stackoverflow.com/}, \url{https://stackexchange.com/}, etc.
	  \item In R: \texttt{?topic}, \texttt{??keyword}
	  \item Examples of general topics: \texttt{?Logic}, \texttt{?Arithmetic}, \texttt{?Syntax}
	  \item Examples of more specific topics: \texttt{?getwd}, \texttt{?dir}
    \end{itemize}
\end{frame}


%% Core material
\subsection{Basic R objects}

\begin{frame}[fragile]
    \frametitle{The environment}
    \begin{itemize}
      \item Working directory, paths, files
      \item libraries and the \emph{global} environment
      \item Objects in the \emph{local} environment
      \item \texttt{options()} function (see documentation for details)
    \end{itemize}
    \verbatiminput{../R/module1/environment.Rout}
\end{frame}


\begin{frame}[fragile]
    \frametitle{Scalars and vectors}
    \verbatiminput{../R/module1/scalars_vectors.Rout}
\end{frame}


\begin{frame}[fragile]
    \frametitle{Matrices}
    \verbatiminput{../R/module1/matrices.Rout}
\end{frame}


\begin{frame}
    \frametitle{Retrieving elements}
    Elements of vectors and matrices can be retrieved either by numeric indexing (1, 2, 3, etc.)
    or by logical indexing (TRUE, FALSE).
    This means that one can index an R object using expressions that resolve to logical vectors.
    In fact, this technique is used a lot in practice.
    Also note that the function \texttt{which} can be used to convert from logical to numeric indexing.
    \begin{itemize}
      \item Vectors
        \begin{itemize}
          \item Retrieve element \texttt{n} of vector \texttt{x}: \texttt{> x[n]}
          \item 3rd element of a vector of length 3:\\ \texttt{> x[c(FALSE, FALSE, TRUE)]}
        \end{itemize}
      \item Matrices (\texttt{r} for row and \texttt{c} for column)
        \begin{itemize}
          \item Retrieve element (\texttt{r}, \texttt{c}) from matrix \texttt{X}: \texttt{> X[r, c]}
          \item Retrieve row \texttt{r} from matrix \texttt{X}: \texttt{> X[r, ]}
          \item Retrieve column \texttt{c} from matrix \texttt{X}: \texttt{> X[, c]}
        \end{itemize}
    \end{itemize}
    \textbf{Question}: What is the difference between \texttt{X[1,2]} and \texttt{X[c(1,2)]}?
\end{frame}


\begin{frame}[fragile]
    \frametitle{Lists}
    \fontsize{11pt}{10}\selectfont
    \verbatiminput{../R/module1/lists.Rout}
\end{frame}


\begin{frame}[fragile]
    \frametitle{Data frames}
    \verbatiminput{../R/module1/dataframes.Rout}
\end{frame}


\begin{frame}[fragile]
    \frametitle{Accessing objects}
	\begin{itemize}
	  \item S3 objects
	    \begin{itemize}
	      \item For vectors and matrices, accessing elements is pretty straightforward (see above)
	      \item For other objects, it can be a little more abstract
	      \item It is easier to understand them through examples (cue mtcars example)
	      \item \texttt{[]}: Access component(s) inside of an object
	      \item \texttt{\$}: Access contents of a single component inside of an object
	      \item \texttt{[[]]}: Same as \texttt{\$}, but more flexible for programming
	    \end{itemize}
	  \item S4 objects are beyond the scope of this course
	\end{itemize}
\end{frame}


\subsection{Control structures}

\begin{frame}[fragile]
    \frametitle{if, else}
    \verbatiminput{../R/module1/if_else.Rout}
\end{frame}


\begin{frame}[fragile]
    \frametitle{for, while}
    \verbatiminput{../R/module1/for_while.Rout}
\end{frame}


\begin{frame}[fragile]
    \frametitle{repeat, next, break}
    \verbatiminput{../R/module1/repeat_next_break.Rout}
\end{frame}


\subsection{Basic object manipulation}

\begin{frame}[fragile]
    \frametitle{c, rbind, cbind}
    \verbatiminput{../R/module1/c_rbind_cbind.Rout}
\end{frame}


\begin{frame}[fragile]
    \frametitle{merge}
    \verbatiminput{../R/module1/merge.Rout}
\end{frame}


\begin{frame}[fragile]
    \frametitle{subset}
    \verbatiminput{../R/module1/subset.Rout}
\end{frame}


\begin{frame}[fragile]
    \frametitle{order, names}
    \verbatiminput{../R/module1/order_names.Rout}
\end{frame}


\subsection{Intermediate object manipulation}

\begin{frame}[fragile]
    \frametitle{aggregate}
    \verbatiminput{../R/module1/aggregate.Rout}
\end{frame}


\begin{frame}[fragile]
    \frametitle{reshape}
    \verbatiminput{../R/module1/reshape.Rout}
\end{frame}


\subsection{Basic variable transformation}

\begin{frame}[fragile]
    \frametitle{Working with dates}
    \verbatiminput{../R/module1/dates.Rout}
\end{frame}


\begin{frame}[fragile]
    \frametitle{Working with times}
    \verbatiminput{../R/module1/times.Rout}
\end{frame}


\begin{frame}
    \frametitle{Useful functions to look up for future advancement}
    We've covered a lot of ground in this module,
    but those interested in more advanced programming in R
    will greatly benefit from researching these functions:\\
    
    \begin{itemize}
      \item \texttt{sum}, \texttt{prod}, and \texttt{ifelse}
      \item \texttt{min}, \texttt{max}, \texttt{pmin}, and \texttt{pmax}
      \item \texttt{mean}, \texttt{var}, and \texttt{cor}
      \item \texttt{nchar}, \texttt{length}, and \texttt{dim}
      \item \texttt{with} and \texttt{within}
      \item \texttt{lapply}, \texttt{sapply}, and other \texttt{*apply}
      \item \texttt{gsub}, \texttt{grep}, \texttt{regexpr}, and other similar ones 
      \item \texttt{Map} and \texttt{Reduce}
      \item \texttt{assign} and \texttt{get}
      \item \texttt{debug} and \texttt{undebug}
    \end{itemize}
\end{frame}


\begin{frame}
    \frametitle{Exercises}
    Take 10 minutes to work on the in-class exercises for module 1.
    They can be found \href{https://github.com/rnitulescu/RcourseOncology2020/blob/master/exercises1.R}{here}.
\end{frame}


%%%%
%% MODULE 2
%%%%

\section{Module 2: Data analysis in R}

\subsection{Basic data analysis}

\begin{frame}[fragile]
    \frametitle{table, xtabs}
    \verbatiminput{../R/module2/table_xtabs.Rout}
\end{frame}


\begin{frame}[fragile]
    \frametitle{summary, sd, var, cor, and quantile}
    \verbatiminput{../R/module2/summary.Rout}
\end{frame}


\subsection{Basic graphing}

\begin{frame}[fragile]
    \frametitle{Scatter plot}
    \fontsize{9}{11}\selectfont
    \verbatiminput{../R/module2/scatter.R} %% Not Rout for this one.
    \begin{figure}[b]
      \centering
      \includegraphics[scale=0.50]{../R/module2/scatter.eps}
    \end{figure}
\end{frame}


\begin{frame}[fragile]
    \frametitle{Quantile-quantile plot}
    \verbatiminput{../R/module2/qqplot.R} %% Not Rout for this one.
    \begin{figure}[b]
      \centering
      \includegraphics[scale=0.50]{../R/module2/qqplot.eps}
    \end{figure}
\end{frame}


\begin{frame}[fragile]
    \frametitle{Histogram and density}
    \verbatiminput{../R/module2/hist_dens.R} %% Not Rout for this one.
    \begin{figure}[b]
      \centering
      \includegraphics[scale=0.50]{../R/module2/hist_dens.eps}
    \end{figure}
\end{frame}


\begin{frame}[fragile]
    \frametitle{Box and whisker plot}
    %\fontsize{9}{11}\selectfont
    \verbatiminput{../R/module2/boxplot.R} %% Not Rout for this one.
    \begin{figure}[b]
      \centering
      \includegraphics[scale=0.50]{../R/module2/boxplot.eps}
    \end{figure}
\end{frame}


\subsection{Basic hypothesis testing}

\begin{frame}[fragile]
    \frametitle{...}
    %\verbatiminput{../R/module2/?.Rout}
\end{frame}

%% under construction ... TODO

\subsection{Ordinary least squares}

%% under construction ... TODO

\subsection{GLM and logistic regression}

%% under construction ... TODO

\subsection{Cox proportional-hazards model}

%% under construction ... TODO


\begin{frame}%% TODO: Exercises for module 2
    \frametitle{Exercises}
    Take 10 minutes to work on the in-class exercises for module 2.
    They can be found \href{https://github.com/rnitulescu/RcourseOncology2020/blob/master/exercises2.R}{here}.
\end{frame}


%%%%
%% MODULE 3
%%%%

\section{Module 3: Research design in R}

%% Power
\subsection{Power and sample size calculations}

\begin{frame}
    \frametitle{The \texttt{pwr} package}
    \begin{itemize}
      \item \texttt{install.packages("pwr")}
      \item \texttt{library(pwr)}
      \item \emph{Statistical Power Analysis for the Behavioral Sciences}, by Jacob Cohen
      \item Example functions: \texttt{pwr.p.test},
            \texttt{pwr.t.test}, \texttt{pwr.2p.test},
            \texttt{pwr.chisq.test}, \texttt{pwr.anova.test}
    \end{itemize}
\end{frame}


\begin{frame}[fragile]
    \frametitle{Power for comparison of two proportions from independent samples}
    \fontsize{10}{12}\selectfont
    \verbatiminput{../R/module3/pwr_2p2n.Rout}
\end{frame}


\begin{frame}[fragile]
    \frametitle{Power for comparison of two means from independent samples}
    \fontsize{10}{12}\selectfont
    \verbatiminput{../R/module3/pwr_t2n.Rout}
\end{frame}


%% under construction ... TODO
%% One example of power/sample size using formula and uniroot() function?


%% Randomizaton
\subsection{Randomization}

\begin{frame}
    \frametitle{Simple randomization}
	Treatment is randomly allocated to each subject according to a pre-determined probability distribution.\\
	
	\bigskip
	
	For example, you plan to recruit 10 subjects and randomly assign to each subject 
	one of two possible treatments, $A$ or $B$, with equal probability.
	
	\bigskip
	
	To implement this randomization scheme, one must only sample, with equal probability,
	10 times from the set $\{A, B\}$ with replacement.
	
	\bigskip
	
	The downside of this method is that it does not guarantee that each treatment group will
	have the same number of subjects (i.e., be balanced).
\end{frame}


\begin{frame}[fragile]
    \frametitle{Simple randomization: Example}
	\verbatiminput{../R/module3/simple_randomization.Rout}
\end{frame}


\begin{frame}
    \frametitle{Replacement randomization}
	Replacement randomization is a method used to ensure that treatment allocations
	done through simple randomization are balanced.
	
	\bigskip
	
	This is put into practice by repeating the simple randomization until balance is achieved.
	
	\bigskip
	
	The downside is that as the sample size and the number of treatment groups increase, the number
	of iterations required to achieve a balanced allocation increases very quickly and can
	become computationally too intensive (this is perhaps not much of an issue nowadays).
	
	\bigskip
	
	A more relevant downside is that there is no guarantee that the treatment allocation will
	consistently remain balanced over time as patients are enrolled and the randomization
	list is followed.
\end{frame}


\begin{frame}[fragile]
    \frametitle{Replacement randomization: Example}
	\verbatiminput{../R/module3/replacement_randomization.Rout}
\end{frame}


\begin{frame}
    \frametitle{Random permuted blocks}
	A solution to the problem of maintaining balanced allocation over time is to use random permuted blocks.

    \bigskip

    A block size is chosen and allocations are permuted within each block.
    This guarantees balance each time a block is completely filled.

    \bigskip

    Ex: \emph{block 1} $[A, A, B, B]$, \emph{block 2} $[B, A, A, B]$, etc.

    \bigskip

    Block sizes can also, optionally, be randomized to make it easier to maintain blinding.
\end{frame}


\begin{frame}[fragile]
    \frametitle{Random permuted blocks: Example}
	\verbatiminput{../R/module3/permuted_blocks.Rout}
\end{frame}


\begin{frame}
    \frametitle{Stratification}
    If one wishes to maintain balance in treatment allocation not only in the overall sample, but also
    in specific sub-groups, then stratified randomization can be used.

    \bigskip

    To perform this kind of randomization, treatments are allocated in such a way to maintain balance across sub-groups.

    \bigskip

    For example, in the context of block randomization, at any time there are two active blocks being filled:
    one for males and one for females. Thus, whenever all blocks are filled, treatment allocation is balanced
    for both males and females.

    \bigskip

    In practice, this simply entails generating one randomization list per sub-group using one of the three
    methods exemplified above.
\end{frame}


\begin{frame}
    \frametitle{Exercises}
    Take 10 minutes to work on the in-class exercises for module 3.
    They can be found \href{https://github.com/rnitulescu/RcourseOncology2020/blob/master/exercises3.R}{here}.
\end{frame}


\begin{frame}
    \frametitle{Review / Questions}
    \vfill
    Does anyone have any specific questions?
    \vfill
\end{frame}


%% END DOCUMENT
\end{document}

